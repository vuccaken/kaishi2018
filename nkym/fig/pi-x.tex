\documentclass[10pt,dvipdfmx]{standalone}
% \documentclass[10pt,dvipdfmx,b5paper,papersize]{jsarticle}
\usepackage{tikz}
\usetikzlibrary{calc}
\usepackage{physics}
% \usetikzlibrary{positioning}


\begin{document}

\begin{tikzpicture}[x=4mm,y=4mm,>=latex]
% \small % 文字サイズ

% gray zone
% \fill[fill=gray!30] (2,0) rectangle (5,5.5);

% 座標系
\draw[->] (0,0) -- (20.5,0) node[right] {$x$} ; % x軸
\draw[->] (0,0) -- (0,8.5) node[above] {$\pi(x)$}; % y軸
% \node(O) at (0,0) [below left]{O}; % 原点
  
% \foreach \x in {0,...,19}{
%   \node at (\x,0) [below] {$\x$};
% }
\node(O) at (0,0) [below]{0}; % 0


% 素数階段
\coordinate (bP) at (0,0);
\coordinate (P) at (0,0);
\foreach \n in {2,3,5,7,11,13,17,19}{
  \path (\n,0) +(bP) coordinate (Q);
  \draw (P) -- ++(Q) -- +(0,1);
  \path (P) ++(Q) +(0,1) coordinate (P);
  \coordinate (bP) at (-\n,0);
  % 飾り
  \draw[dashed] (P) +(0,-1) -- (\n,0);
  \node at (\n,0) [below] {$\n$};
}
\draw (P) -- +(1,0);


\end{tikzpicture}




\end{document}
%
%     巻頭言
%

% 11/14 17:15 nakayama が編集しました。
% 11/21 7:30 nakayama が編集しました。

\documentclass[11pt,b5paper,papersize,dvipdfmx]{jsbook}

\usepackage{vuccaken}
\usepackage{vuccaken2018}
\usepackage{url}


% --------------------------------------
\begin{document}

% \newpage 
% \quad \thispagestyle{empty}
% \newpage
% \quad \thispagestyle{empty}
% \newpage

\clearpage
\thispagestyle{empty}

\begin{hensyukoki}{物理科学科3回生}{中山敦貴}{2018年11月24日}
  今年は去年の反省を踏まえ、会誌の締め切りを10月中に設定していたのに、なぜかまた学園祭前日まで編集作業をしている。というか輪転間に合うのか?
    \footnote{間に合いませんでした。}\par
  前回に続き、今回もまた\LaTeX によって会誌を作成した。今回は表紙から裏表紙まで全て\LaTeX で作成してある。各部員の担当分をまとめる際には、\verb|\input|コマンドの代わりにemath
    \footnote{大熊一弘, \url{http://emath.la.coocan.jp}}
  の\verb|\ReadTeXFile|コマンドを使用させていただいた。非常に便利であり感謝している。\par
  今年の会誌は、全体で100ページを超えることを目標にし、一人20ページ以上というノルマを課していたが、結果として無事100ページ超えを達成することができた。皆さんお疲れ様でした。\par
  まあ皆さん頑張って\LaTeX で書いて提出してくれたのはいいのだが、何度言っても数式に全角アルファベットを使ったり、単位をイタリック体にしてきたりしてきて大変だった。あと個人的には今回初めてTi$k$Zの環境を使ってお絵かきしてみたのだが、かなり満足のいくものができてよかった。皆さんもパワポ卒業しましょう。\par
  もうあと少しで平成は終わってしまうが、とりあえず大阪万博が今日決まったのでよかった。平成が終わると私はもう卒研生なので、平成のうちに量子力学を完全に理解できるよう頑張りたいと思います。
  皆さんも残り少ない平成の時代を充実した日々で送れるよう頑張りましょう。\par
  あと立命は来年の10連休ないらしいです。\par


\vspace{2zw}\noindent
{\bf P.S. 部員へ}\par
  来年度は週4で\LaTeX ゼミやります。残りの3日はMarkdownゼミです。
% \vspace{1zw}

\end{hensyukoki}

% - - - - - - - - - - - - - 

\end{document}
%
% お疲れさまです
%
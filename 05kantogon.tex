%
%     巻頭言
%

% 11/14 19:20 nakayama が編集しました。
% 11/24 22:30 nsmr が編集しました。

\documentclass[11pt,b5paper,papersize,dvipdfmx]{jsbook}

\usepackage{vuccaken}
\usepackage{vuccaken2018}

% --------------------------------------
\begin{document}

% 巻頭言
\begin{kantogon}{物理科学科3回生}{\nsmr}{2018年11月24日}
  こんにちは。物理科学研究会会長の西村宗悟です。今年度は平成最後の年であり、時代が節目を迎える年ですね。節目を迎えるのは年号だけでなく、本研究会も節目の年になります。現在の物理科学研究会の前身である核物理研究会は1949年に発足し、来年で本研究会は70周年を迎えます。今年度と来年度間が節目だからといって、現物理科学研究会の何かが大きく変化するわけでもなく、活動を続けていくと思います。しかし、そういった例年通りの活動できることは大学からの支援やOBの方々からの応援によって初めて実現されることです。支援・応援を受けることが当たり前という考えではなく、支援・応援を受けられることへ感謝しながらこれからも活動を続けていきたいです。\par
  せっかく学外の方に物理科学研究会について知ってもらえる機会なのでこの場で少しどんな活動をしているのかを紹介させていただきます。本研究会では会員それぞれが自分の興味がある分野について研究し、サークル内でそれを発表をしています。分野に関しては物理だけでなく数学や他の応用科学も対象としています。数学や音響学を研究している人もいれば、電子工作やCPUの製作をしている人もいます。そして、本研究会には重大な課題があります。それは会員が少ないということです。毎年、幅広い分野を対象に研究していることを4月の新入生歓迎期に新入生にアピールしているのですが、なかなか入会してくれる人がいません...。今年の新入生歓迎期に研究会宣伝用のビラをある女性に渡そうとしたら彼女に「物理嫌い」と言われました。おそらく本研究会の名前に「物理」という単語が入っていて、「$物理=受験勉強+難しい+面白くない$」というイメージが頭の中で出来上がってしまっているからではないかと思います。そして、彼女の一言を聞いて私は物理の面白さをもっと伝えていきたいと思いました。今年度の学園祭は物理の面白さを伝えることを目標に企画を考えてました!ぜひ、たくさんの人に足を運んでいただきたいです!\par
  さて、本年度の物理科学研究会の会誌「白夜」は過去3年間の中で最もページ数が多くなっています。ページ数が多いから読むのをやめようかな...と思った方もいるのではないでしょうか?ちょっと待ってください!様々な内容があり、難しいものから簡単なものまで幅広く用意したので、まずは軽い気持ちで目を通してみてください!また、セクションで内容が細かく分けられているので、1日で読み切れなくても大丈夫です!会員が数か月かけて製作した会誌なのでぜひ読んでいただきたいです。
  \vspace{-1zw}
\end{kantogon}




\end{document}
%
% お疲れさまです
%
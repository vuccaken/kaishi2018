%
%   hajimeni
%   2018/11/15 22:30
%   2018/11/26
%   

\documentclass[11pt,b5paper,papersize,dvipdfmx]{jsbook}

\usepackage{vuccaken}
\usepackage{vuccaken2018}
\usepackage{12nkym}

\begin{document}

%
\section*{はじめに}
YouTubeで「全ての素数の無限積が$4\pi^2$であることの数学的な証明」
\footnote{
  \url{https://www.youtube.com/playlist?list=PL006ccJyFqlHtQQhonFlgactqdunSQ1OW}
}
なる動画を拝見した。この動画では、無限大に発散してしまうであろう全ての素数の無限積が、なんと有限の値$4\pi^2$に等しくなるという主張をするものであった。しかし、この動画の中での説明では、ゼータ関数の特殊値というものを証明抜きで用いており、それでは納得がいかないので自分で色々と調べてみた。すると、ゼータ関数の解析接続やらで複素解析の知識が必要だったりと結構奥が深いものだったので、まとめるついでに会誌にした。また、ゼータ関数と言えばリーマンであり、リーマンといえばリーマン予想である。せっかくゼータ関数を解析接続するのであるから、ついでにリーマン予想の話も軽くすることにする。\par
\ref{sec:1}節では、動画の内容と同じく、ゼータ関数の特殊値を用いることで
\begin{align}
  2\times 3\times 5\times 7\times 11\times\cdots = 4\pi^2 \notag
\end{align}
という謎の数式が成立してしまうことを示す。ここまでの内容は、上記動画で既に説明されている。
\ref{sec:2}節で複素解析について復習し、\ref{sec:3}節で実際にゼータ関数の解析接続を行い特殊値を計算する。
そして\ref{sec:4}節にて数学界の未解決問題として有名なリーマン予想を\.つ\.い\.で\.に紹介し、最後にゼータ関数の値を数値計算した美しい\zenkakko{?}グラフをいくつか載せることにする。\par
% 10月現在、丁度この記事を書いている時に、微細構造定数の研究の副産物としてリーマン予想が解かれたなどと世間は騒いでいるのだが、結局どうなったのであろうか。それにしてもいいタイミングだったのでこの記事の需要が高まりますな。


%%%
\end{document}